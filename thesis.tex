\documentclass[12pt, notitlepage]{article}
\usepackage[margin=2cm]{geometry}
\usepackage{hyperref}

\title{Functional Testing of Web Applications}
\author{Stefan Gamerith\\\\
		\emph{Linzerstrasse 429 4215,}\\
		\emph{1140 Wien}\\
		\emph{Student ID: 0925081}}

\begin{document}
	\maketitle
	
	\begin{abstract}
		This is done later.
	\end{abstract}
	
	\tableofcontents
	
\section*{Introduction}
Since the first proposal of HTTP\cite{http-proposal} the Web sites evolved from 
text pages implementing the Request/Response Pattern\cite{request-response}  to  
complex Web applications. Whereas the former draws a clear distinction between a client
who performs a request and a server who sends the response, the latter does not show this 
clear differentiation. Even recently a new communication protocol\cite{web-socket} has been published 
in which the browser acts like a server, thus listening for requests. \\
 
Probably one of the most challenging aspects are constant changes.%TO BE CONTINUED% 
Due to the importance of this characteristic a whole section in this thesis outlines the challenges faced
with changing environments. 



All of the above characteristics of Web applications make functional testing a difficult task though not impossible.
This thesis first gives an overview of the different testing approaches with focus on Data-driven approach. For each 
testing method a tool implementing it will be presented. 
%TO BE CONTINUED - more words here%


\end{document}


\bibliography{literature}