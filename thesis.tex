\documentclass[12pt, notitlepage]{article}
\usepackage[margin=2cm]{geometry}
\usepackage{hyperref}

\title{Functional Testing of Web Applications}
\author{Stefan Gamerith\\\\
		\emph{Linzerstrasse 429 4215,}\\
		\emph{1140 Wien}\\
		\emph{Student ID: 0925081}}

\begin{document}
	\maketitle
	
	\begin{abstract}
		This needs to be done.
	\end{abstract}
	
	\tableofcontents
	
\section*{Introduction}
Since the first proposal of HTTP\cite{http-proposal} the Web sites evolved from 
text pages implementing the Request/Response Pattern\cite{request-response}  to  
complex Web applications. Whereas the former draws a clear distinction between a client
who performs a request and a server who sends the response, the latter does not show this 
clear differentiation. Even recently a new communication protocol\cite{web-socket} has been published 
in which the browser acts like a server, thus listening for requests. \\
 
Probably one of the most challenging aspects are constant changes. P.J.Warren, C.Boldyreff, M.Munro \cite{html-evolution}
presented a study which analyzed six websites. Their study shows a correlation between the complexity and the changes of
a website: The more complex a website get the more likely they will change in the future. 
There are structural and behavioral changes. While the former means changes in the document structure (e.g. Document Object Model\cite{dom} or HTML tree), the latter
includes all changes affecting how Web applications response when an action is performed. 
This requires adaptive testing techniques. Due to the importance of this characteristic a whole section in this thesis outlines the challenges faced
with changing environments.\\ 

Another important aspect is the huge user population. While in the beginnings of the World Wide Web only universities had access,
nowadays the vast majority of the population can view content distributed across all over the world. The typical Web user range from a
mother who does an online shopping tour over a teenager updating the current relationship status in Facebook to a student doing research for
the Bachelor thesis. The integration of designers understanding these different social backgrounds of users is crucial. \\


All of the above characteristics of Web applications make functional testing a difficult task though not impossible.
This thesis first gives an overview of the different testing approaches with focus on Data-driven approach. For each 
testing method a tool implementing it will be presented. 
%TO BE CONTINUED - more words here%


\section{Test Automation}
\cite{art-of-software-testing} defines software testing as "the process of executing a program with the intent of finding errors".
According to that one might conclude that finding bugs is the only purpose of software testing. Since TDD\cite{tdd} became popular
another definition coexist: "Software testing is the process of verifying the programs output against predefined values". 
Either way software testing is crucial to ensure quality. \\
Theoretically there are 2 different types of testing technique\cite{art-of-software-testing}:
\begin{itemize}

	\item \textbf{Black-Box Testing}: 

	\item \textbf{White-Box Testing}: 

\end{itemize}

\subsection{Why Test Automation?}

\subsection{Test Automation Frameworks}

\end{document}


\bibliography{literature}