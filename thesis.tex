\documentclass[12pt, notitlepage]{article}
\usepackage[margin=4cm]{geometry}
\usepackage{hyperref}
\usepackage[english]{babel}

\title{Functional Testing of Web Applications}
\author{Stefan Gamerith\\\\
		\emph{Linzerstrasse 429 4215,}\\
		\emph{1140 Wien}\\
		\emph{Student ID: 0925081}}

\begin{document}
	\maketitle
	\thispagestyle{empty}
	\begin{abstract}
		This needs to be done.
	\end{abstract}
	\newpage
	\thispagestyle{empty}
	\tableofcontents
\newpage
\thispagestyle{empty}

\section*{Introduction}
Since the first proposal of HTTP\cite{http-proposal} the Web sites evolved from 
text pages implementing the Request/Response Pattern\cite{request-response}  to  
complex Web applications. Whereas the former draws a clear distinction between a client
who performs a request and a server who sends the response, the latter does not show this 
clear differentiation. Even recently a new communication protocol\cite{web-socket} has been published 
in which the browser acts like a server, thus listening for requests.\\
Probably one of the most challenging aspects are constant changes. P.J.Warren, C.Boldyreff, M.Munro\cite{html-evolution}
presented a study which analyzed six websites. Their study shows a correlation between the complexity and the changes of
a website: The more complex a website get the more likely they will change in the future. 
There are structural and behavioral changes. While the former means changes in the document structure (e.g. Document Object Model\cite{dom} or HTML tree), the latter
includes all changes affecting how Web applications response when an action is performed. 
This requires adaptive testing techniques. Due to the importance of this characteristic a whole section in this thesis outlines the challenges faced
with changing environments.\\ 
Another important aspect is the huge user population. While in the beginnings of the World Wide Web only universities had access,
nowadays the vast majority of the population can view content distributed across all over the world. The typical Web user range from a
mother who does an online shopping tour over a teenager updating the current relationship status in Facebook to a student doing research for
the Bachelor thesis. The integration of designers understanding these different social backgrounds of users is crucial.\\


All of the above characteristics of Web applications make functional testing a difficult task though not impossible.
This thesis first gives an overview of the different testing approaches with focus on Data-driven approach. For each 
testing method a tool implementing it will be presented. 
%TO BE CONTINUED - Brief outline of the contents%
\newpage

\setcounter{page}{1}
\section{Software Testing}
\cite{art-of-software-testing} defines software testing as "the process of executing a program with the intent of finding errors".
According to that one might conclude that finding bugs is the only purpose of software testing. Since TDD\cite{tdd} became popular
another definition coexist: "Software testing is the process of verifying the programs output against predefined values". Besides from 
these IEEE\cite{ieee-testing-definiton} defines software testing as: "The process of operating a system or component under specified conditions, observing or recording the results, and making an evaluation of some aspect of the system or component".\\
\subsection{Testing Techniques}
This section outlines the different software testing techniques\cite{softare-testing-principles}. %TODO more intro
\paragraph{Black-Box Testing} ~\\
This testing technique looks at the code which needs to be tested as a whole. It requires no internal knowledge though test cases are typically 
derived from functional requirements. Therefore some sort of input data is verified against generated outcome.
It is impossible to test all possible input combinations. In \cite{softare-testing-principles} this is named as \textit{Dijkstra's Doctrine}
where for a program accepting a six-character code, ensuring that the first character is numeric and the rest are alphanumeric, an input 
sequence of 9161328320 combinations needs to be generated. 
\paragraph{White-Box Testing} ~\\
In contrast this technique examines the internals such as code, code structure and control flow. White-Box Testing can be further classified in
static and structural testing. The former does not require executing the code thus the source code is sufficient for examination. The source code
is analyzed either by humans through a code review or by static analysis tools which check for unreachable code, unused variables, memory leaks, use
deprecated libraries and other metrics. The latter executes the program and use the control structure for coverage examination\cite{structural-testing}.

\subsection{Why Test Automation?}

\subsection{Test Automation Frameworks}

\bibliography{literature}
\bibliographystyle{plain}
\end{document}


